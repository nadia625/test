\documentclass{report}
\usepackage{ucs}
\usepackage[utf8x]{inputenc}
\usepackage[english,romanian]{babel}
\title{{\sc Raport asupra practicii: 21.06-01.07.2016}}
\author{Prenume NUME}
\date{\,}
\begin{document}
\maketitle

\tableofcontents

\chapter{Introducere}

Acest raport se poate și trebuie scris folosind modelul unei lucrări de licență care se găsește pe site-ul facultății. 

\vskip 0.5cm

Raportul asupra practicii efectuate zilnic intre datele 26.06-07.07.2016. 

\chapter{Activități planificate}
\begin{enumerate}
\item  Luni, 26.06.2017 \newline
Aducerea la cunoștință a obiectivelor și cerințelor practicii de producție
\item  Marți, 27.06.2017 \newline
Configurarea sistemelor software pe calculatoare. 
\item  Miercuri, 28.06.2017 \newline
Studierea modului de lucru cu Git. Interfețe grafice de lucru cu Git (SmartGit).
\item  Joi, 29.06.2016 \newline
Studierea și practicarea LaTeX
\item  Vineri, 30.06.2017  \newline
Inițierea unei lucrări (descrierea unui algoritm, a unei teme agreate cu prof. coordonator)
\item  Luni, 03.07.2017  \newline
Lucrul asupra lucrării
\item  Marți, 04.07.2017  \newline
Lucrul asupra lucrării
\item  Miercuri, 05.07.2017  \newline
Prezentarea lucrărllor
\item  Joi, 06.07.2017  \newline
Prezentarea lucrărilor
\item  Vineri, 07.07.2017  \newline
Notarea finală a activității
\end{enumerate}
\chapter{26.06.2017}
Am desfăţurat următoarele activităţi:
\begin{itemize}
\item
Am identificat sursele pentru MikTeX, Git, SmartGit și BitBucket.
\begin{itemize}
\item
Am identificat sursele pentru MikTeX, Git, SmartGit și BitBucket.
\item
Am instalat, configurat pe calculatorul de lucru aplicațiile necesare:
\begin{itemize}
\item
MikTeX
\item
SmartGit
\item
Bitbucket
\end{itemize}
\item
Am instalat, configurat pe calculatorul de lucru aplicațiile necesare:
\begin{itemize}
\item
MikTeX
\item
SmartGit
\item
Bitbucket
\end{itemize}

\end{itemize}
\end{itemize}

\chapter{26.06.2017}
Studierea obiectivelor și cerințelor față de practica de producție. Clarificarea situațiilor incerte.
\chapter{27.06.2017}
Am studiat modul de lucru cu Git și interfața grafică de lucru cu Git (SmartGit).
\chapter{28.06.2017}
Am studiat și am practicat Latex.
\chapter{29.06.2017}
Am inițiat o lucrare scrisă în Latex.
\chapter{30.06.2017}
Am continuat lucrul asupra temei alese.
\chapter{03.07.2017}
Am continuat lucrul asupra temei și am terminat .
\chapter{04.07.2017}
Am continuat lucrul asupra temei și am terminat .
Prezentarea proiectului.
\chapter{05.07.2017}
Prezentarea proiectului.
\chapter{06.07.2017}
Prezentarea proiectului.
\chapter{07.07.2017}

Notarea finală a activității.

\chapter{Concluzii}
Am invățat să lucrez cu Latex ,Git și BitBucket. Aici pot fi prezentate succint lucrurile învățate (câte ceva despre git, latex, bitbucket, etc - se pot adăuga secțiuni pentru fiecare subiect). 

Și acum să cităm unele dintre referințele noastre \cite{DUMMY:1} și \cite{book:25008}. La fel putem să cităm și alte cărți și surse online cum ar fi \cite{book:776133, book:1045183}. Alte exemple sunt în mostra / modelul unei lucrări de licență de pe site-ul facultății. 

% aici urmeaza declaratiile care fac posibila includerea bibliografiei in format bibtex. 

\bibliography{referinte} 
\bibliographystyle{ieeetr}

\end{document}
